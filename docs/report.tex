\documentclass[a4paper, 12pt]{article}
\setlength{\oddsidemargin}{0in} \setlength{\evensidemargin}{0in}
\setlength{\textwidth}{6.2in}
\setlength{\topmargin}{-0.2in} \setlength{\textheight}{8.8in}

\usepackage{graphicx}

\title{CS22510 Building occupancy grids}
\author{Michal Wojciech Goly [mwg2]}
\date{18th March 2016}

\begin{document}

\maketitle

\section{Design and implementation}
\subsection{Initial approach}
I approached this assignment slightly differently than I usually do.
Because I have never programmed in object oriented C++ before, instead of
jumping straight into writing code, I begun by going through online
tutorials to quickly learn the new concepts. I realised that because this
assignment was relatively small, I could rapidly create a working prototype
without using any of the object oriented features. 

I took a piece of paper and tried to calculate the coordinates of the first
obstacle myself, before hard coding it into my program. I wrote some utility
functions, which could be used for all the conversions and maths necessary 
to calculate the position of the obstacle in metres, along with the conversion
mechanism to indices. I took into account the 0.1 metres offset in the grid,
and after successfully calculating the first occupied cell, I moved on to 
reading data from the source files provided. I decided to store both poses
and ranges in an appropriate 2D vectors of doubles. I also stored sensor 
angles in a separate vector, which made it possible to iterate over all the
vectors and calculating obstacle positions in the grid. I would ignore 
any range of 2.5 metres or more, as it referred to infinity reading of a 
sensor in the assignment specification. Finally, I marked occupied cells as
asterisks in the previously created matrix of chars initialised with spaces. 

\subsection{Object oriented solution}
Even though the prototype described above did the job, I wanted to take advantage
of the object oriented features of C++. I also thought about making my 
approach of marking cells a bit smarter by using probability values, instead of
binary true or false. Finally, I wanted to make the presentation a little bit
nicer to look at, therefore I decided to use the allegro5\cite{1} library to 
create the GUI. 

I identified 3 classes I could split my code into:
\begin{itemize}
	\item \texttt{GridFrame} which would take care of the GUI and initialisation
		of the other objects in the system.
	\item \texttt{Grid} which would represent the 50x50 grid in the system.
	\item \texttt{Robot} which would hold the information about the poses and 
		ranges readings, along with the sensor angles data.
\end{itemize}

I also decided to move all the utility functions into a separate namespace
\texttt{converter}, because I have done some research on the Internet, and 
apparently there is no point in making a class with only static member
functions in C++. Because it is a multi-paradigm language I thought a separate
namespace made more sense. 

As mentioned before, I tried to use probability values instead of spaces and 
asterisks to mark potentially occupied cells. I decided that all the cells will
start with a probability of 5. This value could be then decreased or increased 
as required by appropriate functions later in the program, up to the minimum
and maximum values of 1 and 9 respectively. After each subsequent reading, I
wanted to increase the probability of the cell containing the obstacle, and decrease
the probability of all the cells between the robot and the current obstacle. 
I achieved that functionality using the Bresenham line drawing algorithm\cite{2}, which
I modified to create a vector of pairs of indices of all the cells between
two cells specified.

Finally, I really enjoyed using the allegro5 library to create the GUI of the 
program. I managed to create a 500x500 pixels window with a title, presented
in an appropriate position on the screen. In order to create an animation, I
paint the grid within the window for each row in the input files. Different 
shades of grey colour are used to indicate the probability of given cell being
occupied. White means cell is unlikely to be occupied, whereas black means there
is a very high possibility given cell is occupied. I also use an appropriate
event queue in order to make sure the user is able to close the window whenever
he wants to. Animation stops at the end and waits for the user to exit the
window.  

Overall I am happy with my solution to this assignment. I have learnt quite
a lot about C++ and I am sure this knowledge will help me during the subsequent
assignment.

\newpage
\addcontentsline{toc}{section}{REFERENCES}
\begin{thebibliography}{2}
\bibitem{1} http://liballeg.org/ [Accessed: 14th March 2016]
\bibitem{2} http://tech-algorithm.com/articles/drawing-line-using-bresenham-algorithm/ [Accessed: 14th March 2016]

\end{thebibliography}

\end{document}

